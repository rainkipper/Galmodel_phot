\documentclass{article}
\begin{document}
	\section{Yhikud} % (fold)
	\label{sec:yhikud}
	\begin{tabular}{llll}
		Suurus & input yhik & Sisemine yhik & olemus\\
		\hline
		$X_i$, $Y_i$ & pix & pix & sisuliselt reaalarvuline piksli koordinaat maatriksil\\
		$X_p$, $Y_p$ & arcsec & radiaan & piltide omavahelise yhendamise koordinaadistiks\\
		$X_c$, $Y_c$ & kpc & kpc & komponendi koordinaadid, mis viidud juba kaugele\\
		$d$ & kpc & kpc & komponendi kaugus meist\\
		$Rz$ & kpc & kpc & profiili koordinaadid \\
		$\theta$ & deg & rad & nurk silindrilistest koordinaatides\\
	\end{tabular}
	% section yhikud (end)
	
	
	
	
	\section{Kokkulepped} % (fold)
	\label{sec:kokkulepped}
	\begin{enumerate}
		\item Profiili yks parameetritest on alati $M$, mis korreleerub 1:1 heledusega... selle abil skaleerin komponendi piltide omavahelist kokkupanekut
		\item Model-image on moeldud ainult yhe komponendi pildina... ehk siis model-image-id on kokku piltide arv * komponentide arv
		\item piksli koordinaat on piksli keskkoha koordinaat
		\item kui voetakse teise komp oma, siis tuleb otse viidata, mitte kolmandate kaudu (ehk $1->2$ mitte $1->3$ $3->2$)
		\item Kui maski v22rtus $>0.5$, siis kasutatakse fittimisel
		\item esialgu on koik sisend uniform prior
	\end{enumerate}

	% section kokkulepped (end)
	
	
	\section{Varia} % (fold)
	\label{sec:varia}
	
	N2idisgalaktika on TKRS 5661, mille punanihe 0.528501 lum dist 2936Mpc
	
	\begin{equation}
		L\left[ 10^{10}L_\odot \right] = N\cdot10^{\left[ 0.4(\mathrm{ZP}-M_\odot) -14 + 2\log(d[\mathrm{kpc}]) \right]},
	\end{equation}
	kus $d$ on heleduskaugus
	% section varia (end)
	
	
	\section{TODO} % (fold)
	\label{sec:todo}
	\begin{enumerate}
		\item Kui mitme kompnendi vahelist seost vaja ning komponendid nihkes yksteiseset, siis tuleb koik info sisse anda phys koordinaatides ning teisendada igal komponendil eraldi.... aeganoudev
		\item PSF
		\item loglike juurde eraldi masside fittimine, mis eraldab masside fittimise multinesti fittimisest.
		\item yhikute kordaja igal komponendil erinev
	\end{enumerate}
	% section todo (end)
\end{document}